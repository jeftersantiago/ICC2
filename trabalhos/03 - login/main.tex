% Created 2022-12-10 Sat 16:45
% Intended LaTeX compiler: pdflatex
\documentclass[11pt]{article}
\usepackage[utf8]{inputenc}
\usepackage[T1]{fontenc}
\usepackage{graphicx}
\usepackage{longtable}
\usepackage{wrapfig}
\usepackage{rotating}
\usepackage[normalem]{ulem}
\usepackage{amsmath}
\usepackage{amssymb}
\usepackage{capt-of}
\usepackage{hyperref}
\usepackage{tikz}
\author{Jefter Santiago}
\date{\today}
\title{}
\hypersetup{
 pdfauthor={Jefter Santiago},
 pdftitle={},
 pdfkeywords={},
 pdfsubject={},
 pdfcreator={Emacs 28.2 (Org mode 9.5.5)}, 
 pdflang={English}}
\begin{document}

\tableofcontents

/**
   Jefter Santiago Mares
   nºUSP: 12559016
 **/
\#include "HashTable.h"
\#include "Student.h"


char * readLine();
int main () \{

int n;
scanf("\%d\n", \&n);

HashTable * table = makeHashTable(n);
Student * student = NULL;
int i;

i = 0;
while(i < n) \{
  student = newStudent(readLine());
  if(insertHashTable(table, student))
    printf("Cadastro efetuado com sucesso\n");
  else
    printf("NUSP ja cadastrado\n");
  i++;
\}
scanf("\%d\n", \&n);
i = 0;
while(i < n)\{

    Student * student = NULL;
    char * c =  readLine();
    int k = loginHashTable(table, c, \&student);
    if(k == 0)\{
      printGrades(\&student);
    \}
    else if (k == 2)\{
      printf("NUSP invalido\n");
    \}
    else if (k == 1)\{
      printf("Senha incorreta para o NUSP digitado\n");
    \}
    free(c);
    c = NULL;
    i++;
  \}
  deleteHashTable(table);
  return 0;
\}


char *readLine() \{
  char *string = NULL;
  char currentInput;
  int index = 0;
  do \{
    currentInput = (char)getchar();
    string = (char *) realloc(string, sizeof(char) * (index + 1));
    string[index] = currentInput;
    index++;
    if(currentInput == '\r')
      currentInput = (char)getchar();
  \} while((currentInput != '\n') \&\& (currentInput != EOF));
  string[index - 1] = '$\backslash$0';
  return string;
\}
\end{document}